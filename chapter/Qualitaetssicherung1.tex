\subsection{Techniken zur Validierung und Qualitätssicherung}
\subsubsection{Definition von Qualitätskriterien}
Um eine hochwertige Anforderungsanalyse zu erstellen, ist es wichtig, dass nachfolgend aufgeführte Qualitätsmerkmale eingehalten werden.
Ansonsten kann es zu Problemen kommen, die sich später im Projekt wiederfinden.

Typische Qualitätskriterien für Anforderungsanalysen sind folgende:
\begin{itemize}
    \item \textbf{Abgestimmt:} Die Anforderungen sind für alle Stakeholder als notwendig und korrekt akzeptiert
    \item \textbf{Eindeutig:} Die Anforderungen sind widerspruchsfrei dokumentiert
    \item \textbf{Notwendig:} Die Anforderungen haben Gültigkeit für alle Stakeholder
    \item \textbf{Konsistent:} Die Anforderungen sind widerspruchsfrei und können überprüft werden
    \item \textbf{Realisierbar:} Umsetzung der Anforderungen aus organisatorischer, rechtlicher, technischer und finanzieller Perspektive möglich
    \item \textbf{Verfolgbar:} Der Anforderungsursprung und -umsetzung sowie die Beziehung zu anderen Dokumenten sind nachvollziehbar
    \item \textbf{Vollständig:} Alle Anforderungen sind dokumentiert
    \item \textbf{Verständlich:} Möglichkeit der Umsetzung aus organisatorischer, rechtlicher, technischer und finanzieller Perspektive
\end{itemize}\autocite[vgl.][Seite 44]{Maulhardt.b}

Mit der Hilfe dieser Qualitätskriterien kann die Anforderungsanalyse systematisch validiert werden.
Um diese Validierung durchführen zu können, sollten insbesondere die drei nachfolgenden Hauptziele der Überprüfung betrachtet werden:
\begin{itemize}
    \item \textbf{Inhalt:} Es wurden alle relevanten Anforderungen im erforderlichen Detailgrad erfasst
    \item \textbf{Dokumentation:} Die Anforderungen wurden nach den vorgegebenen Richtlinien dokumentiert
    \item \textbf{Abgestimmtheit:} Alle Stakeholder haben die Anforderungen akzeptiert
\end{itemize}
Anforderungen, die nicht nach den Hauptzielen überprüft wurden, sollten nicht für die Entwicklung freigegeben werden\autocite[vgl.][Seite 16ff]{Maulhardt.c}.

\paragraph{Inhalt}
Dieser Qualitätsaspekt beschreibt inhaltliche Fehler der Anforderungen.
Somit ist dieser Qualitätsaspekt gegeben, sobald bei der Überprüfung der Anforderungen mit den nachfolgenden Kriterien keine Fehler gefunden werden:
\begin{itemize}
    \item Vollständigkeit sowohl auf alle Anforderungen als auch auf einzelne Anforderungen bezogen
    \item Verfolgbarkeit
    \item Korrektheit/Adäquatheit
    \item Konsistenz
    \item Keine vorzeitigen Entwurfsentscheidungen
    \item Überprüfbarkeit
    \item Notwendigkeit
\end{itemize}\autocite[vgl.][Seite 19f]{Maulhardt.c}

\paragraph{Dokumentation}
Dieser Qualitätsaspekt beschreibt die Fehler, die bei der Dokumentation der Anforderungen auftreten können.
So kann eine Missachtung von Richtlinien oder eine unklare Formulierung zu Fehlern führen, welche erst in der Produktentwicklung auftreten.

Prüfkriterien für die Dokumentation sind:
\begin{itemize}
    \item Konformität zum Dokumentationsformat und zur Dokumentenstruktur
    \item Verständlichkeit
    \item Eindeutigkeit
    \item Konformität mit Dokumentationsregeln
\end{itemize}\autocite[vgl.][Seite 21ff]{Maulhardt.c}

\paragraph{Abgestimmtheit}
Dieser Qualitätsaspekt beschreibt die Fehler, die bei der Abstimmung der Anforderungen zwischen den relevanten Stakeholdern auftreten können.
So erfüllt die Anforderungsanalyse dieses Qualitätsmerkmal, wenn die Anforderungen abgestimmt sind und alle potenziellen Konflikte aufgelöst wurden\autocite[vgl.][Seite 24]{Maulhardt.c}.

\subsubsection{Prinzipien zur Überprüfung der Qualitätskriterien}
Um die Anforderungen auf die bereits definierten Qualitätsmerkmale zu überprüfen, gibt es verschiedene Prinzipien, die in der Praxis angewendet werden.

%Beteiligung der richtigen Stakeholder
Zuerst ist es wichtig, die richtigen Stakeholder in den Prozess einzubinden.
Je nach Ziel der Prüfung ist eine unterschiedliche Auswahl an Stakeholdern sinnvoll.
Allerdings ist es in jedem Fall wichtig, dass die Prüfer unabhängig sind und sowohl interne (also direkt am Projekt beteiligte) als auch externe (also nicht direkt am Projekt beteiligte) Stakeholder involviert sind\autocite[vgl.][Seite 26]{Maulhardt.c}.

%Trennung von Fehlersuche und Fehlerkorrektur
Ein weiteres Prinzip ist die Trennung von der Suche nach Fehlern und der Korrektur dieser Fehler.
So sollen Mängel bei der Überprüfung lediglich dokumentiert werden.
Anschließend können diese Mängel in einem zweiten Schritt nach der Überprüfung korrigiert werden\autocite[vgl.][Seite 26]{Maulhardt.c}.

%Prüfung aus unterschiedlichen Sichten
Ähnlich wie bei Gerichtsverhandlungen kann es Sinn machen, die Anforderungen aus unterschiedlichen Sichten zu betrachten.
So kann die Anforderungsanalyse aus der Sicht des Kunden, des Entwicklers und des Anwenders betrachtet werden, ähnlich wie vor Gericht eine Tat aus verschiedenen Perspektiven geschildert wird.
Dadurch können Fehler, die aus einer Perspektive nicht sichtbar sind, aufgedeckt werden\autocite[vgl.][Seite 27]{Maulhardt.c}.

%Geeigneter Wechsel der Dokumentationsform
Da verschiedene Dokumentationsformen unterschiedliche Stärken und Schwächen haben, kann es sinnvoll sein, die Dokumentationsform zu wechseln.
Natürliche Sprache ist sehr gut darin, eine hohe Ausdruckskraft zu erreichen, jedoch ist sie schlecht darin, komplexe Sachverhalte zu beschreiben.
Da können Diagramme oder Tabellen eine bessere Alternative sein.
Durch diese Abwechslung können Schwächen in der einen Form durch eine andere ausgeglichen werden\autocite[vgl.][Seite 28]{Maulhardt.c}.

\subsubsection{Prüfungstechniken}
Nachdem die Qualitätskriterien sowie die Prinzipien für die Überprüfung der Anforderungen definiert wurden, können nun die verschiedenen Prüfungstechniken betrachtet werden.
Diese beschreiben, wie die konkrete Überprüfung auf die bereits definierten Qualitätskriterien erfolgen soll.

Manuelle Prüftechniken, auch Reviews genannt, können in drei verschiedene Arten unterteilt werden:
\begin{itemize}
    \item \textbf{Stellungnahme:} Der Prüfer (eine dritte Person wie z.B. ein Kollege) liest die Anforderungen und überprüft diese nach den Qualitätsmängeln
    \item \textbf{Inspektion:} Aufteilung in vier Phasen:
    \begin{enumerate}
        \item Planung: Definition der Prüfziele und Teilnehmer
        \item Übersicht: Autor beschreibt die zu prüfenden Anforderungen
        \item Fehlersuche: Inspektoren suchen entweder allein oder in Gruppen nach Fehlern
        \item Fehlersuche und -konsolidierung: Die gefundenen Fehler werden besprochen und dokumentiert
    \end{enumerate}
    \item \textbf{Walkthrough:} Der Autor sowie die Reviewer suchen in den Anforderungen nach Qualitätsmängeln.
        Anschließend stellt der Autor den Reviewern die Anforderungen vor, um noch zusätzliche Informationen zu vermitteln und ein konsolidiertes Verständnis der Anforderungen zu erreichen.
\end{itemize}\autocite[vgl.][Seite 32ff]{Maulhardt.c}

Weitere mögliche Prüfungstechniken sind das perspektivenbasierte Lesen (Überprüfung aus verschiedenen Perspektiven), die Prüfung durch Prototypen sowie der Einsatz von Checklisten\autocite[vgl.][Seite 33]{Maulhardt.c}.

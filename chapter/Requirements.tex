\subsection{Erstellung des Requirements-Katalog}\label{subsec:requirements}
Unter Anwendung der bereits erläuterten Methoden und unter Verwendung der genannten Tools wurde im Rahmen dieser Arbeit
schließlich ein Requirements-Katalog erstellt (siehe~\nameref{subsec:requirements_catalouge}).
Hier nur ein paar kurze Nutzungsinformationen für den Katalog.
Da das in \textbf{ReqView} zur Erstellung des Katalogs verwendete Template in Englisch verfasst ist, wurde aus Gründen
der Konsistenz der gesamte Katalog in Englisch verfasst.
Die Anforderungen sind außerdem nicht nach Priorität oder den Kano-Kategorien sortiert, sondern folgen einer durch das
Template vorgegebenen Hierarchie.
Es handelt sich des Weiteren um die finale Version des Katalogs, das heißt Konflikte zwischen Anforderungen wurden bereits
gelöst (siehe~\nameref{subsubsec:conflicts}) und sind damit nicht mehr ersichtlich.

\subsubsection{Konflikte zwischen Anforderungen}\label{subsubsec:conflicts}
Während der Erstellung des Requirements-Katalogs kam es zu einigen Konflikten zwischen Anforderungen, deren Lösungen im Folgenden
fachgerecht beschrieben werden sollen.
Zentral hierfür ist die bereits gezeigte Matrix der Konsolidierungstechniken (siehe~\nameref{fig:konsolidierungstechniken}).

\paragraph{Verschiedene Abonnements}
Ursprünglich war geplant eine Anwendung zu entwickeln, welche nur ein mögliches Abonnement anbietet.
Nach der Kategorisierung der Anforderungen nach dem Kano-Modell durch die verschiedenen Stakeholder wurden allerdings
große Unterschiede zwischen den Basisfaktoren und den Leistungsfaktoren festgestellt.
Aufgrund der vielen verschiedenen Meinungen, der geplanten langen Lebensdauer des Ergebnisses und da dieser Konflikt die
Sachebene betrifft, wurde unter Anwendung der Variantenbildung entschieden mehrere Abonnements anzubieten.

\paragraph{Gezielte Werbung}
Um den Gewinn der Anwendung zu erhöhen hat die Unternehmensführung des entwickelnden Unternehmens entschieden in der PWA
optionale Cookies für gezielte Werbung zu verwenden.
Dies stieß bei den Kunden/Nutzern auf Ablehnung.
Da bei diesem Fall die Eindeutigkeit einer Lösung wichtig war und auch in diesem Fall mit einer langen Lebensdauer des
Ergebnisses gerechnet wurde, kam es zwischen den beteiligten zu der Einigung, dass Werbung nur bei Nutzern mit normalen
Accounts geschaltet wird.

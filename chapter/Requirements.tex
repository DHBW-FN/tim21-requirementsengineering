\subsection{Erstellung des Requirements-Katalog}\label{subsec:requirements}
Die Erstellung des Anforderungskatalogs erfolgte nach der in~\nameref{sec:vorgehensweise} beschriebenen Vorgehensweise.
Erster Schritt war dabei die Erhebung der Anforderungen mithilfe der in~\nameref{subsec:anforderungsermittlung} beschriebenen
Erhebungsmethoden.
Die theoretische Natur und das damit verbundene nicht Existieren eines Großteils der Stakeholder führte dabei dazu, dass
Befragungstechniken nur eine begrenzte Nützlichkeit aufwiesen.
Daher wurden überwiegend Kreativtechniken sowie dokumenten- und systembasierte Techniken angewandt.
Erste angewandte Technik war dabei das Brainstorming (Kreativtechnik).
Dadurch konnte eine erste grobe Liste von hauptsächlich funktionalen Anforderungen erstellt werden.
Ergänzt wurden diese durch das Anwenden eines Perspektivwechsels (Kreativtechnik), bei dem sich in die Rolle der nicht existierenden Stakeholder
hineinversetzt wurde, um die Anwendung so aus deren Sicht zu betrachten.
Da auch hierbei überwiegend funktionale Anforderungen aufkamen, wurde für die Erhebung der nicht-funktionalen Anforderungen
auf die dokumenten- und systembasierte Techniken zurückgegriffen.
Dazu wurden verwandte Anwendungen, in diesem Fall \textit{Prime Video} und \textit{Netflix}, sowohl auf ihre Funktionen als auch
soweit möglich auf deren technische Voraussetzungen untersucht.
Das Ergebnis dieser Untersuchungen war eine umfangreiche Liste von funktionalen und nicht-funktionalen Anforderungen.
Diese Liste wurde anschließend nach den in~\nameref{subsec:techniken-zur-validierung-und-qualitatssicherung} beschriebenen
Methoden systematisch auf ihre Qualität und Validität untersucht.
Anschließend erfolgte die Kategorisierung nach den in~\nameref{subsec:kano-modell} beschriebenen Kano-Kategorien.
Darauf basiert die Einordnung der Anforderungen in \textbf{Kann-Anforderungen} und \textbf{Muss-Anforderungen}
(siehe~\nameref{subsubsec:structure}).
Letzter Schritt war die Identifizierung und fachgerechte Lösung von Anforderungskonflikten wie es in~\nameref{subsubsec:conflicts}
näher beschrieben ist.
Ergebnis dieses Vorgehens war ein Requirements-Katalog, welcher, aus Gründen der Übersichtlichkeit dieser Arbeit, im Anhang
zu finden ist (siehe~\nameref{subsec:requirements_catalouge}).
Die Struktur des Katalogs wird zusätzlich in~\nameref{subsubsec:structure} näher beschrieben um den Umgang mit diesem zu
erleichtern.

\subsubsection{Struktur des Anforderungskatalogs}\label{subsubsec:structure}
Um den Umgang mit dem im Anhang befindlichen Anforderungskatalogs zu vereinfachen, wird hier kurz die Struktur erläutert.
Zunächst sei gesagt, dass da das in \textbf{ReqView} zur Erstellung des Katalogs verwendete Template in Englisch verfasst ist,
aus Gründen der Konsistenz der gesamte Katalog in Englisch verfasst wurde.
Der Katalog hat die Form einer Tabelle mit den folgenden Spalten:
\begin{itemize}
    \item \textbf{Type}
    \item \textbf{Description}
    \item \textbf{Source}
    \item \textbf{Priority}
    \item \textbf{Verification Method}
    \item \textbf{Kano-Model Category}
\end{itemize}

\paragraph{Type} beschreibt um was für eine Art Element es sich handelt.
Die verwendeten Typen sind:
\begin{itemize}
    \item \textbf{Section:} Ein Element das zur Strukturierung des Katalogs dient
    \item \textbf{Information:} Ein Element, das Informationen über die Anwendung oder relevante äußere Faktoren/Entitäten
    enthält
    \item \textbf{Functional Requirement:} Eine funktionale Anforderung
    \item \textbf{Non-Functional Requirement:} Eine nicht-funktionale Anforderung
\end{itemize}

\paragraph{Description} enthält eine Nummer, einen Titel und eine kurze aber genaue Beschreibung des Elements.
Die Nummer wird verwendet, um die Elemente zu strukturieren und zu sortieren und dient außerdem dem Aufzeigen einer Hierarchie
zwischen den Elementen.

\paragraph{Source} enthält den/die Stakeholder der/die diese Anforderung erhoben haben.
Daher wird dieses Attribut nicht bei den Elementen vom Typ \textbf{Section} und \textbf{Information} verwendet.

\paragraph{Priority} priorisiert die Anforderung nach der Wichtigkeit für die Anwendung in \textbf{hoch}, \textbf{mittel}
und \textbf{niedrig} und gibt damit eine grobe Auskunft in welcher Reihenfolge die Anforderungen umgesetzt werden.
Auch dieses Attribut wird nicht bei den Elementen vom Typ \textbf{Section} und \textbf{Information} verwendet.

\paragraph{Verification Method} beschreibt die Methode, die verwendet wird um eine erfolgreiche Umsetzung der Anforderung
zu überprüfen.
Wird nur bei den Elementen vom Typ \textbf{Functional Requirement} und \textbf{Non-Functional Requirement} verwendet.

\paragraph{Kano-Model Category} dient der Kategorisierung der Anforderung nach den fünf Kano-Kategorien.
Diese Kategorisierung führt zudem zur Trennung der Anforderungen in \textbf{Muss-Anforderungen} und \textbf{Kann-Anforderungen},
die wie folgt definiert ist:

\begin{itemize}
    \item \textbf{Muss-Anforderungen:} Basisfaktoren (Threshold), Leistungsfaktoren (Performance)
    \item \textbf{Kann-Anforderungen:} Begeisterungsfaktoren (Excitement), Gleichgültige Qualitäten (Indifferent),
    Umgekehrte Qualitäten (Reverse)
\end{itemize}

Wichtig ist zusätzlich, dass es sich bei dem vorliegendenKatalog um die finale Version handelt, das heißt eventuelle Konflikte
zwischen Anforderungen wurden bereits identifiziert und gelöst (siehe~\nameref{subsubsec:conflicts}) und sind damit nicht
mehr ersichtlich.

\subsubsection{Konflikte zwischen Anforderungen}\label{subsubsec:conflicts}
Während der Erstellung des Requirements-Katalogs kam es zu einigen Konflikten zwischen Anforderungen, deren Lösungen im Folgenden
fachgerecht beschrieben werden sollen.
Zentral hierfür ist die in~\nameref{subsubsec:loesen} erläuterte Reihenfolge der Widerspruchsauflösung und die bereits
gezeigte Matrix der Konsolidierungstechniken (siehe~\nameref{fig:konsolidierungstechniken}).

\paragraph{Ungenaue Anforderungen an die Performance (Konsistenzkonflikt)}
Durch Stakeholder ohne technisches Wissen wurden wiederholt vage Anforderungen an die Performance gestellt, wie etwa
\enquote{die Anwendung muss schnell sein}.
Anstatt diese Anforderungen sofort detaillierter zu beschreiben, wurden diese stattdessen in den Katalog übernommen, unter
der Annahme, dass andere Beteiligte sich um das detailliertere Beschreiben kümmern würden.
Dadurch kam es in der ersten Version des Anforderungskatalogs zu Konsistenzkonflikten.
Behoben werden konnten diese durch ein Anforderungsreview unter spezieller Beteiligung der Entwickler, welche aufgrund des
tieferen technischen Wissens Empfehlungen für die detaillierte Beschreibung der Anforderungen geben konnten.

\paragraph{Gezielte Werbung (Anforderungskonflikt)}
Ein weiterer Konflikt drehte sich um die Frage, ob in der Anwendung Werbung gezeigt werden sollte oder nicht.
Entstanden ist dieser, da die Geschäftsleitung des entwickelnden Unternehmens ihren Plan Werbung in der Applikation zu zeigen
vor den übrigen Stakeholdern geheim gehalten hat.
Dadurch kam es zu einem Anforderungskonflikt, da die Nutzer und Kunden explizit eine werbefreie Anwendung wünschten.
Nach einer Analyse des Konflikts zeigte sich, dass es sich sowohl um einen Sachkonflikt als auch einen Interessenkonflikt
handelt.
Die Einordnung als Sachkonflikt erfolgte, da den übrigen Stakeholdern durch die Geschäftsleitung bewusst Informationen
vorenthalten wurden.
Die Einordnung als Interessenskonflikt erfolgte, da die verschiedenen Anforderungen durch die subjektiven Interessen der
beteiligten Stakeholder geprägt waren.
Für diesen Konflikt war es, um die Rentabilität der Anwendung für das Unternehmen zu gewährleisten, notwendig eine eindeutige
Lösung zu finden.
Außerdem war zu beachten, dass das Ergebnis eine lange Lebensdauer haben würde.
Deshalb wurde unter Beachtung des Sachkonfliktaspekts und unter Vernachlässigung des Interessenkonfliktaspekts entschieden
eine Einigung zwischen den Stakeholdern anzustreben.
Erreicht werden konnte diese durch das Zugeständnis der Geschäftsleitung, dass die Werbung nur für Profile, welche das Standardabonnement
abgeschlossen haben gezeigt wird, was von den übrigen Stakeholdern akzeptiert wurde.

\paragraph{Erheben von Daten (Anforderungskonflikt)}
Durch das steigende Bewusstsein für die Privatsphäre und die Sicherheit der eigenen Daten, wünschten sich Nutzer und Kunden
eine Anwendung, welche abgesehen von den zur Registrierung erforderlichen Daten keine weiteren erhebt.
Da zur zukünftigen Weiterentwicklung der Anwendung von den Entwicklern und der Geschäftsleitung Nutzungsdaten gefordert wurden,
kam es hier zu einem Konflikt.
Zusätzlich entstand durch nationale und internationale Gesetze die Notwendigkeit, den Nutzer zu lokalisieren um potenziell
verbotene Inhalte sperren zu können.
Eine Analyse des Konflikts ergab die Einteilung als Interessenkonflikt und als struktureller Konflikt.
Die Einteilung als Interessenkonflikt erfolgte erneut, da die Anforderungen durch die subjektiven Interessen der Stakeholder
geprägt waren.
Die Beteiligung (inter-) nationaler Behörden führte zur Einteilung als struktureller Konflikt, da diese gegenüber den
Nutzern und der entwickelnden Organisation über eine höhere Macht und Autorität verfügen.
An diesem Konflikt war eine große Anzahl an Stakeholdern beteiligt.
Außerdem war er geprägt von einer hohen Kritikalität des Sachverhaltes.
Ebenfalls gab es zwischen den beteiligten Stakeholdern ein starkes Machtgefälle.
Da durch den Konflikt sowohl die Interessensebene als auch die Strukturebene betroffen war und eine Abstimmung kein zufriedenstellendes
Ergebnis erbrachte, wurde entschieden diesen Konflikt nach dem Ober sticht unter Prinzip zu lösen.
Ergebnis dieser Lösung war, dass durch die Geschäftsleitung der Organisation entschieden wurde, dass nicht nur die für die
Einhaltung (inter-)nationaler Gesetze nötigen Daten erhoben werden, sondern zusätzlich auch die von den Entwicklern geforderten
Nutzungsdaten.
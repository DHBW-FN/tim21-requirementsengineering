\subsection{Erstellung des Requirements-Katalog}\label{subsec:requirements}
Unter Anwendung der in~\nameref{subsubsec:anforderungsermittlung} beschriebenen Methoden für die Erhebung von Anforderungen
wurde anschließend ein Requirements-Katalog erstellt.
Besondere Bedeutung kamen dabei den Kreativtechniken und den dokumenten- und systembasierten Techniken zu, da diese ohne
Beteiligung verschiedener Stakeholder auskommen.
Die erhobenen Anforderungen wurden anschließend mithilfe der in~\nameref{subsec:techniken-zur-validierung-und-qualitatssicherung}
erläuterten Qualitätskriterien und deren Prüfungstechniken auf ihre Qualität und Validität untersucht.
Das führte zu einem finalen Ergebnis welches aus Gründen der Übersichtlichkeit im Anhang zu finden ist
(siehe~\nameref{subsec:requirements_catalouge}).
Zusätzlich ein paar kurze Nutzungsinformationen für den Katalog.
Da das in \textbf{ReqView} zur Erstellung des Katalogs verwendete Template in Englisch verfasst ist, wurde aus Gründen
der Konsistenz der gesamte Katalog in Englisch verfasst.
Die Anforderungen sind außerdem nicht nach Priorität oder den Kano-Kategorien sortiert, sondern folgen einer durch das
Template vorgegebenen Hierarchie.
Die Unterscheidung zwischen Kann- und Muss-Anforderungen erfolgt implizit anhand der Einordnung in die Kano-Kategorien und
ist wie folgt aufgeteilt:

\begin{itemize}
    \item \textbf{Muss-Anforderungen:} Basisfaktoren (Threshold), Leistungsfaktoren (Performance)
    \item \textbf{Kann-Anforderungen:} Begeisterungsfaktoren (Excitement), Gleichgültige Qualitäten (Indifferent),
    Umgekehrte Qualitäten (Reverse)
\end{itemize}

Es handelt sich des Weiteren um die finale Version des Katalogs, das heißt Konflikte zwischen Anforderungen wurden bereits
gelöst (siehe~\nameref{subsubsec:conflicts}) und sind damit nicht mehr ersichtlich.

\subsubsection{Konflikte zwischen Anforderungen}\label{subsubsec:conflicts}
Während der Erstellung des Requirements-Katalogs kam es zu einigen Konflikten zwischen Anforderungen, deren Lösungen im Folgenden
fachgerecht beschrieben werden sollen.
Zentral hierfür ist die in~\nameref{subsubsec:loesen} erläuterte Reihenfolge der Widerspruchsauflösung und die bereits
gezeigte Matrix der Konsolidierungstechniken (siehe~\nameref{fig:konsolidierungstechniken}).

\paragraph{Ungenaue Anforderungen an die Performance (Konsistenzkonflikt)}
Durch Stakeholder ohne technisches Wissen wurden wiederholt vage Anforderungen an die Performance gestellt, wie etwa
\enquote{die Anwendung muss schnell sein}.
Anstatt diese Anforderungen sofort detaillierter zu beschreiben, wurden diese stattdessen in den Katalog übernommen, unter
der Annahme, dass andere Beteiligte sich um das detailliertere Beschreiben kümmern würden.
Dadurch kam es in der ersten Version des Anforderungskatalogs zu Konsistenzkonflikten.
Behoben werden konnten diese durch ein Anforderungsreview unter spezieller Beteiligung der Entwickler, welche aufgrund des
tieferen technischen Wissens Empfehlungen für die detaillierte Beschreibung der Anforderungen geben konnten.

\paragraph{Gezielte Werbung (Anforderungskonflikt)}
Ein weiterer Konflikt drehte sich um die Frage, ob in der Anwendung Werbung gezeigt werden sollte oder nicht.
Entstanden ist dieser, da die Geschäftsleitung des entwickelnden Unternehmens ihren Plan Werbung in der Applikation zu zeigen
vor den übrigen Stakeholdern geheim gehalten hat.
Dadurch kam es zu einem Anforderungskonflikt, da die Nutzer und Kunden explizit eine werbefreie Anwendung wünschten.
Nach einer Analyse des Konflikts zeigte sich, dass es sich sowohl um einen Sachkonflikt als auch einen Interessenkonflikt
handelt.
Die Einordnung als Sachkonflikt erfolgte, da den übrigen Stakeholdern durch die Geschäftsleitung bewusst Informationen
vorenthalten wurden.
Die Einordnung als Interessenskonflikt erfolgte, da die verschiedenen Anforderungen durch die subjektiven Interessen der
beteiligten Stakeholder geprägt waren.
Für diesen Konflikt war es, um die Rentabilität der Anwendung für das Unternehmen zu gewährleisten, notwendig eine eindeutige
Lösung zu finden.
Außerdem war zu beachten, dass das Ergebnis eine lange Lebensdauer haben würde.
Deshalb wurde unter Beachtung des Sachkonfliktaspekts und unter Vernachlässigung des Interessenkonfliktaspekts entschieden
eine Einigung zwischen den Stakeholdern anzustreben.
Erreicht werden konnte diese durch das Zugeständnis der Geschäftsleitung, dass die Werbung nur für Profile, welche das Standardabonnement
abgeschlossen haben gezeigt wird, was von den übrigen Stakeholdern akzeptiert wurde.

\paragraph{Erheben von Daten (Anforderungskonflikt)}
Durch das steigende Bewusstsein für die Privatsphäre und die Sicherheit der eigenen Daten, wünschten sich Nutzer und Kunden
eine Anwendung, welche abgesehen von den zur Registrierung erforderlichen Daten keine weiteren erhebt.
Da zur zukünftigen Weiterentwicklung der Anwendung von den Entwicklern und der Geschäftsleitung Nutzungsdaten gefordert wurden,
kam es hier zu einem Konflikt.
Zusätzlich entstand durch nationale und internationale Gesetze die Notwendigkeit, den Nutzer zu lokalisieren um potenziell
verbotene Inhalte sperren zu können.
Eine Analyse des Konflikts ergab die Einteilung als Interessenkonflikt und als struktureller Konflikt.
Die Einteilung als Interessenkonflikt erfolgte erneut, da die Anforderungen durch die subjektiven Interessen der Stakeholder
geprägt waren.
Die Beteiligung (inter-) nationaler Behörden führte zur Einteilung als struktureller Konflikt, da diese gegenüber den
Nutzern und der entwickelnden Organisation über eine höhere Macht und Autorität verfügen.
An diesem Konflikt war eine große Anzahl an Stakeholdern beteiligt.
Außerdem war er geprägt von einer hohen Kritikalität des Sachverhaltes.
Ebenfalls gab es zwischen den beteiligten Stakeholdern ein starkes Machtgefälle.
Da durch den Konflikt sowohl die Interessensebene als auch die Strukturebene betroffen war und eine Abstimmung kein zufriedenstellendes
Ergebnis erbrachte, wurde entschieden diesen Konflikt nach dem Ober sticht unter Prinzip zu lösen.
Ergebnis dieser Lösung war, dass durch die Geschäftsleitung der Organisation entschieden wurde, dass nicht nur die für die
Einhaltung (inter-)nationaler Gesetze nötigen Daten erhoben werden, sondern zusätzlich auch die von den Entwicklern geforderten
Nutzungsdaten.
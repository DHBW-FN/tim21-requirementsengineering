\externaldocument{anhang}
\subsection{Ereignisgesteuerte Prozesskette (EPK)}\label{subsec:epk)}
Ereignisgesteuerte Prozessketten sind seit ihrer Entwicklung im Jahr 1992 eine beliebte und weitverbreitete Methode zur 
Geschäftsprozessmodellierung (vgl.~\autocite{epc-diagram}).
Grund hierfür ist die geringe Zeit, die für die Erstellung dieser Diagramme benötigt wird.
Des Weiteren sind EPK-Modelle übersichtlich und leicht verständlich.
Damit eignen sie sich ideal um im Rahmen des Requirements Engineering die Kernprozesse einer Anwendung zu modellieren.
Anhand dieser Modelle können anschließend Anforderungen an die Anwendung abgeleitet werden.
Aus diesen Gründen wurden im Rahmen dieser Arbeit einige EPK-Modelle erstellt (siehe~\nameref{subsec:epk-modelle}), um den Prozess der Anforderungserstellung zu unterstützen.
Da es keine eindeutig zentralisierte schreibweise für EPK-Modelle gibt, wurde ebenfalls eine Legende der verwendeten Konventionen beigefügt (siehe~\nameref{fig:epk_legende}).

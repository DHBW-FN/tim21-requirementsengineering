\section{Durchführung der Anforderungsanalyse}\label{sec:durchfuhrung-der-anforderungsanalyse}

\subsection{Prozessmodellierung}\label{subsec:prozessmodelle}
Um das Verständnis für den gegebenen Use-Case und die darauf basierende Anforderungsanalyse zu steigern soll nun einer der
mithilfe eines EPK-Modells dargestellten Prozesse beispielhaft näher erläutert werden.
Zusätzlich soll erklärt werden, inwiefern ein solches Modell den Vorgang der Erhebung und Definition von Anforderungen
unterstützt.
Bei dem Prozess infrage handelt es sich um den in~\nameref{fig:epk_search} dargestellten Prozess zum Finden und Abspielen
eines speziellen Films.

\subsubsection{Prozessbeschreibung}\label{subsubsec:prozessbeschreibung}
Der Prozess startet mit der Ankunft des Nutzers auf der Startseite der Anwendung.
Hier soll er die Auswahl haben nach einem bestimmten Film zu suchen oder einen Film aus den ihm vorgeschlagenen Filmen
auszuwählen und zu starten.
Entscheidet er sich für Letzteres, so wird der ausgewählte Film gestartet und der Prozess endet.
Entscheidet er sich hingegen für die Suche nach einem bestimmten Film, so kann er diese durchführen und erhält eines von
zwei Ergebnissen.
Wird der Film in der Datenbank gefunden, soll er dem User angezeigt werden.
Dieser hat dann wiederum die Wahl auf die Startseite zurückzukehren (Prozess beginnt von Neuem), den Film abzuspielen
(Prozess endet) oder einen anderen Film zu suchen (letzter Prozessschritt wird wiederholt).
Wird der Film nicht gefunden, soll der Nutzer informiert werden, dass der Film nicht vorhanden ist und automatisch auf die
Startseite zurückgeleitet werden.

\subsubsection{Nutzen für die Anforderungserhebung}\label{subsubsec:nutzen}
Anhand eines solchen sehr simplen Prozessmodells können sehr schnell und einfach erste grobe Anforderungen erkannt werden,
die im weiteren Verlauf der Anforderungsanalyse weiter ausspezifiziert werden können.
Dabei dienen sowohl Zustände als auch Ereignisse als Quelle.
Aus Zuständen lassen sich beispielsweise Anforderungen an die Benutzeroberfläche ableiten oder Outputs von Ereignissen
erkennen.
In diesem Fall abgeleitet werden, dass die Anwendung eine Startseite braucht, auf der der Nutzer nach dem Login ankommt.
Außerdem kann eine Suche nach einem Film entweder erfolgreich oder erfolglos sein, das heißt es muss eine entsprechende
Reaktion für beide Fälle geben.
Anhand von Ereignissen hingegen lassen sich Anforderungen im Hinblick auf Interaktionen des Users mit dem System oder
automatisch ablaufenden Funktionen des Systems selbst ableiten.
So kann aus den Ereignissen in diesem Prozessmodell abgeleitet werden, dass dem Nutzer auf der Startseite einige Filme
vorgeschlagen werden sollen, die er dann direkt starten können soll.
Zudem soll es möglich sein nach Filmen zu suchen, wofür ein Input Feld zur Verfügung stehen muss.
Wichtig ist im Umgang mit solchen Prozessmodellen zu beachten, dass die daraus abgeleiteten Anforderungen häufig noch zu
ungenau formuliert sind, um sie direkt in den Anforderungskatalog mit aufzunehmen.
Zusätzlich können aus den Prozessmodellen nur sehr selten nicht-funktionale Requirements abgeleitet werden.
Aus diesen Gründen sollte die Verwendung dieser Modelle unbedingt durch die Anwendung weiterer Methoden ergänzt werden.
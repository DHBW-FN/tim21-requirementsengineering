\subsection{Widersprüche der Anforderungen}\label{subsec:widersprueche}
\gqq{Zwei Anforderungen widersprechen einander, wenn sie nicht durch dieselbe technische Lösung umgesetzt werden können.}\autocite[][S.233]{Herrmann.2022}
So definiert Herrmann Widersprüche.
Widersprüche können sich aus unterschiedlichen Gründen ergeben.
So können Widersprüche zum Beispiel in Bezug auf die Qualität, die Kosten oder die Zeit entstehen.

In diesem Kapitel wird erläutert, wie Widersprüche erkannt, analysiert und gelöst werden können.

\subsubsection{Erkennen von Widersprüchen}\label{subsubsec:erkennung}
Bevor Widersprüche gelöst werden können, müssen sie zuerst erkannt werden.
Es gibt verschiedene Indikatoren, die auf Widersprüche hinweisen können.
\begin{itemize}
    \item Bisher getroffene Aussagen werden ignoriert oder verändert, so als wären diese nie getroffen worden.
    \item Blindes Zustimmen zu oder Ablehnen von Aussagen anderer.
    \item Pedanterie
    \item Aussagen anderer werden bis ins kleinste Detail hinterfragt.
    \item Informationen oder Informationsdetails werden verheimlicht.
    \item Man lässt sich nur auf vage Aussagen ein, mit der Aufforderung an andere, diese zu detaillieren.
\end{itemize}~\autocite[vgl.][S.43]{OliverCreighton.2012}

\subsubsection{Analyse von Widersprüchen}\label{subsubsec:analyse}
Bevor Widersprüche gelöst werden können, müssen diese analysiert werden.
Dazu werden die Ursachen für den Widerspruch ermittelt.
Auch kann man Widersprüche in verschiedene Kategorien einteilen.

Es kann zwischen drei Arten von Widersprüchen unterschieden werden:
\begin{itemize}
    \item Inkonsistenz
    \item Anforderungskonflikte
    \item Machbarkeitskonflikte
\end{itemize}~\autocite[vgl.][S.235f]{OliverCreighton.2012}

\paragraph{Inkonsistenz}


\paragraph{Anforderungskonflikt}


\paragraph{Machbarkeitskonflikt}


\subsubsection{Lösen von Widersprüchen}\label{subsubsec:loesen}

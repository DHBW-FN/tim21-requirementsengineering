\subsection{Widersprüche der Anforderungen}\label{subsec:widersprueche}
\gqq{Zwei Anforderungen widersprechen einander, wenn sie nicht durch dieselbe technische Lösung umgesetzt werden können.}\autocite[][S.233]{Herrmann.2022}
So definiert Herrmann Widersprüche.
Widersprüche können sich aus unterschiedlichen Gründen ergeben.
So können Widersprüche zum Beispiel in Bezug auf die Qualität, die Kosten oder die Zeit entstehen.

In diesem Kapitel wird erläutert, wie Widersprüche erkannt, analysiert und gelöst werden können.

\subsubsection{Erkennen von Widersprüchen}\label{subsubsec:erkennung}
Bevor Widersprüche gelöst werden können, müssen sie zuerst erkannt werden.
Es gibt verschiedene Indikatoren, die auf Widersprüche hinweisen können.
\begin{itemize}
    \item Bisher getroffene Aussagen werden ignoriert oder verändert, so als wären diese nie getroffen worden.
    \item Blindes Zustimmen zu oder Ablehnen von Aussagen anderer.
    \item Pedanterie
    \item Aussagen anderer werden bis ins kleinste Detail hinterfragt.
    \item Informationen oder Informationsdetails werden verheimlicht.
    \item Man lässt sich nur auf vage Aussagen ein, mit der Aufforderung an andere, diese zu detaillieren.
\end{itemize}~\autocite[vgl.][S.43]{OliverCreighton.2012}

\subsubsection{Analyse von Widersprüchen}\label{subsubsec:analyse}
Bevor Widersprüche gelöst werden können, müssen diese analysiert werden.
Dazu werden die Ursachen für den Widerspruch ermittelt.
Auch kann man Widersprüche in verschiedene Kategorien einteilen.

Es kann zwischen drei Arten von Widersprüchen unterschieden werden:
\begin{itemize}
    \item Inkonsistenz
    \item Anforderungskonflikte
    \item Machbarkeitskonflikte
\end{itemize}~\autocite[vgl.][S.235f]{OliverCreighton.2012}

\paragraph{Inkonsistenz}
Ein Anforderungswiderspruch vom Typ Inkonsistenz bezieht sich auf Inkonsistenzen zwischen verschiedenen Anforderungen,
die sich auf den Problemraum beziehen.
Dies können beispielsweise Inkonsistenzen in der Terminologie, unklare Formulierungen,
fehlende Informationen, falsche Inhalte oder mehrdeutige Aussagen sein.
Das bedeutet, dass die Anforderungen untereinander in Konflikt stehen, aber keine technischen Probleme aufweisen.
Inkonsistenzen können durch Anforderungsreviews entdeckt werden, bei denen Anforderungen thematisch gruppiert werden,
um festzustellen, ob sie widersprüchliche Aussagen enthalten.
Inkonsistenzen werden durch eine Entscheidung im Problemraum gelöst,
zum Beispiel durch eine Klärung der Terminologie oder durch eine Überarbeitung der Formulierungen.\autocite[vgl.][S.235]{Herrmann.2022}

\paragraph{Anforderungskonflikt}
Ein Anforderungskonflikt ist ein Widerspruchstyp,
bei dem zwei oder mehr Anforderungen an ein System oder einen Prozess miteinander in Konflikt stehen.
Das bedeutet, dass es nicht möglich ist, alle Anforderungen gleichzeitig zu erfüllen,
da sie sich gegenseitig widersprechen oder unvereinbar sind.
Anforderungen können aus verschiedenen Quellen stammen, wie zum Beispiel Kundenanforderungen,
gesetzliche Vorschriften oder technische Einschränkungen.
Ein Beispiel für einen Anforderungskonflikt könnte sein,
dass ein System sowohl extrem sicher als auch sehr einfach zu bedienen sein soll.
Diese beiden Anforderungen sind miteinander in Konflikt,
da zusätzliche Sicherheitsmaßnahmen in der Regel die Benutzerfreundlichkeit beeinträchtigen können.
Anforderungskonflikte können in der Planungs- und Entwicklungsphase eines Systems oder Prozesses auftreten
und erfordern oft eine sorgfältige Abwägung und Priorisierung der verschiedenen Anforderungen.\autocite[vgl.][S.235f]{Herrmann.2022}

\paragraph{Machbarkeitskonflikt}
Ein Machbarkeitskonflikt ist ein Widerspruchstyp,
bei dem zwei oder mehr Anforderungen an ein System oder einen Prozess in Konflikt geraten,
da sie nicht gleichzeitig erfüllt werden können,
ohne dass es zu einem Verlust von Effizienz, Qualität oder Sicherheit kommt.
In diesem Konflikt geht es darum, dass die Anforderungen gegenseitig die Machbarkeit beeinträchtigen.

Ein Beispiel dafür ist der Konflikt zwischen der Notwendigkeit, ein System sehr sicher zu machen,
und der Notwendigkeit, es sehr effizient zu gestalten.
Eine sehr sichere Methode kann beispielsweise zusätzliche Überprüfungen und Genehmigungen erfordern,
was zu längeren Wartezeiten führt und somit die Effizienz beeinträchtigen kann.
Umgekehrt kann eine sehr effiziente Methode möglicherweise notwendige Sicherheitschecks vernachlässigen
und damit die Sicherheit des Systems beeinträchtigen.\autocite[vgl.][S.236]{Herrmann.2022}

\subsubsection{Lösen von Widersprüchen}\label{subsubsec:loesen}

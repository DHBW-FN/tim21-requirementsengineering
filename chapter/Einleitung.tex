\section{Einleitung}
\subsection{Aufgabenstellung}
Im Rahmen dieser Hausarbeit soll eine Anforderungsanalyse für ein Softwareprojekt durchgeführt werden.
Dabei soll ein Video on Demand System entwickelt werden, welches es ermöglicht, Videos über das Internet anzusehen.
Dies soll in Anlehnung an die Webseite \url{https://www.primevideo.com} erfolgen.
Einschränkend ist dabei zu erwähnen, dass ausschließlich aktuelle Kinofilme angeboten werden sollen.
Ältere Filme oder Serien sollen somit nicht vom System angeboten werden.

\subsection{Aufbau der Arbeit}
Dieses Dokument ist in zwei Teile aufgeteilt.
Im ersten Teil wird auf die wissenschaftliche Herangehensweise an die Anforderungsanalyse eingegangen.
Hierzu werden die einzelnen Aspekte einer Anforderungsanalyse beschrieben und die einzelnen Methoden, welche in der Anforderungsanalyse verwendet werden, erklärt.
Dies wird mit wissenschaftlichen Quellen untermauert, um die allgemeine Qualität dieser Anforderungsanalyse zu gewährleisten.

Der zweite Teil beschreibt die praktische Umsetzung der Anforderungsanalyse des Video on Demand Systems.
Auf Basis der in Teil 1 beschriebenen Methoden wird Schritt für Schritt erklärt, wie die Ergebnisse der Anforderungsanalyse entstanden sind.
Der Fokus hierbei liegt auf den erarbeiteten Anforderungen, auf welche im Text zwar eingegangen wird, jedoch nicht im Detail.
Der gesamte Anforderungskatalog befindet sich im Anhang, um eine bessere Übersichtlichkeit zu gewährleisten.
Auf diese Weise kann der Leser die Anforderungen nachlesen, ohne den Text zu stören.

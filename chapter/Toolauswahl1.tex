\section{Vorgehensweise zur Erstellung der Anforderungsanalyse}\label{sec:vorgehensweise}

\subsection{ReqView als Tool zur Verwaltung von Anforderungen}\label{sec:tool}
Zuerst muss geklärt werden, in welcher Form die Anforderungen verwaltet werden sollen.
Hierzu stehen eine Menge von Tools zur Verfügung.
Für dieses Projekt wird das Tool \textbf{ReqView} verwendet.

Dieses Tool ist mit einer Lizenz für Studenten kostenlos und kann unter \url{https://www.reqview.com/} entweder für alle gängigen Plattformen heruntergeladen werden oder im Internet direkt verwendet werden.
Dadurch geschieht keine Einschränkung bezüglich der Plattform, auf der die Anforderungen verwaltet werden sollen.
Insbesondere die vorgefertigten Vorlagen, welche eine Anforderungsanalyse nach ISO/IEC/IEEE 29148:2018 ermöglichen, sind ein Grund für die Wahl dieses Tools.

Auch die Versionierung der Anforderungen ist mit diesem Tool möglich.
Dies ist wichtig, da die Anforderungen im Laufe des Projekts verändert werden können.
So kann nachvollzogen werden, welche Anforderungen zu welcher Version gehören.

Ebenso ist die Integration mit Jira ein wichtiger Punkt.
Jira ist ein Projektmanagement-Tool, welches im Projekt verwendet wird.
Deshalb ist es wichtig, dass die Anforderungen, welche aus der Anforderungsanalyse entstehen, direkt in Jira importiert werden können.
Auch die Möglichkeit, den aktuellen Projektfortschritt wieder in ReqView zu importieren, führt zu einer besseren Übersichtlichkeit.

Zuletzt ist die Möglichkeit, die Anforderungen in verschiedenen Formaten zu exportieren, ein wichtiger Punkt.
So können die Anforderungen in verschiedenen Formaten, wie z.B. Excel, Word oder HTML, exportiert werden.
Diese verschiedenen Formate können dem Kunden zur Verfügung gestellt werden, um die Anforderungen zu prüfen und eine strukturierte Anforderungsanalyse zu erhalten\autocite[vgl.][]{eccam.2023}.
